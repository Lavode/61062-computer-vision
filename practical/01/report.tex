\documentclass[a4paper]{scrreprt}

% Uncomment to optimize for double-sided printing.
% \KOMAoptions{twoside}

% Set binding correction manually, if known.
% \KOMAoptions{BCOR=2cm}

% Localization options
\usepackage[english]{babel}
\usepackage[T1]{fontenc}
\usepackage[utf8]{inputenc}

% Quotations
\usepackage{dirtytalk}

% Floats
\usepackage{float}

% Enhanced verbatim sections. We're mainly interested in
% \verbatiminput though.
\usepackage{verbatim}

% Automatically remove leading whitespace in lstlisting
\usepackage{lstautogobble}

% PDF-compatible landscape mode.
% Makes PDF viewers show the page rotated by 90°.
\usepackage{pdflscape}

% Advanced tables
\usepackage{array}
\usepackage{tabularx}
\usepackage{longtable}

% Fancy tablerules
\usepackage{booktabs}

% Graphics
\usepackage{graphicx}

% Current time
\usepackage[useregional=numeric]{datetime2}

% Float barriers.
% Automatically add a FloatBarrier to each \section
\usepackage[section]{placeins}

% Custom header and footer
\usepackage{fancyhdr}

\usepackage{geometry}
\usepackage{layout}

% Math tools
\usepackage{mathtools}
% Math symbols
\usepackage{amsmath,amsfonts,amssymb}
\usepackage{amsthm}
% General symbols
\usepackage{stmaryrd}

% Utilities for quotations
\usepackage{csquotes}

% Bibliography
\usepackage[
  style=alphabetic,
  backend=biber, % Default backend, just listed for completness
  sorting=ynt % Sort by year, name, title
]{biblatex}
\addbibresource{references.bib}

\DeclarePairedDelimiter\abs{\lvert}{\rvert}
\DeclarePairedDelimiter\norm{\lVert}{\rVert}
\DeclarePairedDelimiter\floor{\lfloor}{\rfloor}

\DeclareMathOperator{\sgn}{sgn}

% Bullet point
\newcommand{\tabitem}{~~\llap{\textbullet}~~}

\pagestyle{plain}
% \fancyhf{}
% \lhead{}
% \lfoot{}
% \rfoot{}
% 
% Source code & highlighting
\usepackage{listings}

% SI units
\usepackage[binary-units=true]{siunitx}
\DeclareSIUnit\cycles{cycles}

\newcommand{\lecture}{61062 - Computer Vision}
\newcommand{\series}{01}
% Convenience commands
\newcommand{\mailsubject}{\lecture - Practical \series}
\newcommand{\maillink}[1]{\href{mailto:#1?subject=\mailsubject}
                               {#1}}

% Should use this command wherever the print date is mentioned.
\newcommand{\printdate}{\today}

\subject{\lecture}
\title{Practical \series}
\subtitle{Report}

\author{Michael Senn \maillink{michael.senn@students.unibe.ch} --- 16-126-880}

\date{\printdate}

% Needs to be the last command in the preamble, for one reason or
% another. 
\usepackage{hyperref}

\begin{document}
\maketitle


\setcounter{chapter}{\numexpr \series - 1 \relax}

\chapter{Practical - Report}

For the whole report, assume that we are working on a $m \times n$ image with
the first dimension being its width. We define indices the way they are
commonly defined for 2D matrices, starting at $(0, 0)$ in the top left, the
first coordinate indicating the row, the second coordinate indicating the
column.

As an example, a $4 \times 3$ image would have the following pixels:
\[
		\begin{bmatrix}
				a_{0, 0} & a_{0, 1} & a_{0, 2} & a_{0, 3} \\
				a_{1, 0} & a_{1, 1} & a_{1, 2} & a_{1, 3} \\
				a_{2, 0} & a_{2, 1} & a_{1, 2} & a_{2, 3}
		\end{bmatrix}
\]

\section{Derivative of the data term}
\label{sec:derivative_of_data_term}

Starting with the provided discretization of the data term:
\[
		\abs*{u \times k - g}^2 = \sum_{i=0}^{m-1} \sum_{j=0}^{n-1} \abs*{g[i, j] - \sum_{p=0}^{1} \sum_{q=0}^{1} k[p, q] \cdot u[i - p + 1, j - q + 1]}_2^2
\]

We simplify the data term by substituting the assigned blur kernel $k_0$:
\[
		k_0 \coloneqq \begin{bmatrix}
				\frac{1}{2} & \frac{1}{2} \\
				0 & 0
		\end{bmatrix}
\]

\begin{align*}
		D[u] & \coloneqq \abs*{u \times k_0 - g}^2 \\ 
			 & = \sum_{i=0}^{m-1} \sum_{j=0}^{n-1} \abs*{g[i, j] - \frac{1}{2} \cdot (u[i + 1, j + 1] + u[i + 1, j])}_2^2 \\
			 & = \sum_{i=0}^{m-1} \sum_{j=0}^{n-1} \left(g[i, j] - \frac{1}{2} \cdot (u[i + 1, j + 1] + u[i + 1, j])\right)^2
\end{align*}

Where the second simplification is due to the $l_2$ norm of a scalar being
equal to the absolute value of a scalar --- and the square of a real value
being equal to its absolute square. We denote this kernel-specific data term as
$D[u]$ to abbreivate notation. Let further $f(i, j)$ denote the summand:

\begin{align*}
		f(i, j) & \coloneqq \left(g[i, j] - \frac{1}{2} \cdot (u[i + 1, j + 1] + u[i + 1, j])\right)^2
\end{align*}

\subsection{Cases for partial derivation}

We now observe where factors of $u[i, j]$ appear in the expansion of the sum
$D[u]$, to determine the different cases relevant for the differentiation. Note
that there are four possible cases, visualized in figure
\ref{fig:data_term_derivatives}

\begin{figure}
		\centering
		\includegraphics[width=0.6\textwidth]{resources/data_term_derivatives.png}
		\caption{Regions of derivatives of data term}
		\label{fig:data_term_derivatives}
\end{figure}


\subsubsection{Case \emph{A}}

For $i = 0$, none of the factors of the sum will take on the form $u[i, j]$, so
the derivative is a constant $0$:

\[
		\nabla_{D, A} \coloneqq 0
\]


\subsubsection{Case \emph{B}}

For $1 \leq i \leq n-1$, $j = 0$, the factors of the sum will take on the form
$u[i, j]$ for $f(i - 1, j)$. As such:

\begin{align*}
		\nabla_{D, B} \coloneqq \frac{\partial D[u]}{\partial u[i, j]} = \frac{\partial f(i - 1, j)}{\partial u[i, j]}
\end{align*}


\subsubsection{Case \emph{C}}

For $1 \leq i \leq n-1$, $j = m - 1$, the factors of the sum will take on the form
$u[i, j]$ for $f(i - 1, j - 1)$. As such:

\begin{align*}
		\nabla_{D, C} \coloneqq \frac{\partial D[u]}{\partial u[i, j]} = \frac{\partial f(i - 1, j - 1)}{\partial u[i, j]}
\end{align*}


\subsubsection{Case \emph{D}}

For $1 \leq i \leq n-1$, $1 \leq j \leq m - 2$, the factors of the sum will
take on the form $u[i, j]$ for $f(i - 1, j)$ and $f(i - 1, j - 1)$. As such:

\begin{align*}
		\nabla_{D, D} \coloneqq \frac{\partial D[u]}{\partial u[i, j]} = \frac{\partial f(i - 1, j)}{\partial u[i, j]} + \frac{\partial f(i - 1, j - 1)}{\partial u[i, j]}
\end{align*}

\subsection{Determining the derivatives}

We now determine the partial derivatives for $u[i, j]$ of $f(i - 1, j - 1)$ and
$f(i - 1, j)$, using basic laws for derivatives.

\begin{align*}
		\frac{\partial f(i - 1, j)}{\partial u[i, j]} & =
		\frac{1}{2} \cdot u[i, j + 1] + \frac{1}{2} \cdot u[i, j] - g[i - 1, j] \\
		\frac{\partial f(i - 1, j - 1)}{\partial u[i, j]} & =
		\frac{1}{2} \cdot u[i, j - 1] + \frac{1}{2} \cdot u[i, j] - g[i - 1, j - 1]
\end{align*}

Then the derivatives for the four cases above follow directly:

\subsubsection{Case \emph{A}}

\[
		\nabla_{D, A} = 0
\]

\subsubsection{Case \emph{B}}

\[
		\nabla_{D, B} = 
		\frac{1}{2} \cdot u[i, j + 1] + \frac{1}{2} \cdot u[i, j] - g[i - 1, j],
\]

\subsubsection{Case \emph{C}}

\[
		\nabla_{D, C} = 
		\frac{1}{2} \cdot u[i, j - 1] + \frac{1}{2} \cdot u[i, j] - g[i - 1, j - 1]
\]

\subsubsection{Case \emph{D}}

\[
		\nabla_{D, D} = 
		  \frac{1}{2} \cdot u[i, j + 1] + \frac{1}{2} \cdot u[i, j] - g[i - 1, j] + \frac{1}{2} \cdot u[i, j - 1] + \frac{1}{2} \cdot u[i, j] - g[i - 1, j - 1]
\]

\section{Derivative of Gaussian prior}
\label{sec:derivative_of_gaussian}

Starting with the provided discretization of the Gaussian prior, adjusted for a
$m \times n$ image:

\begin{align*}
		G[u] & \coloneqq \sum_{i=0}^{n-2} \sum_{j=0}^{m-2} \left((u[i + 1, j] - u[i, j])^2 + (u[i,j + 1] - u[i, j])^2\right) \\
			 & + \sum_{i=0}^{n-2} \left((u[i + 1, m - 1] - u[i, m - 1])^2\right) \\
			 & + \sum_{j=0}^{m-2} \left((u[n - 1, j + 1] - u[n - 1, j])^2\right)
\end{align*}

Let $f(\hat{i}, \hat{j})$ denote the summand of $G[u]$ for $i = \hat{i}, j =
\hat{j}$.  That is:

\begin{align*}
		f(i, j) \coloneqq 
		\begin{cases}
				(u[i + 1, j] - u[i, j])^2 + (u[i, j + 1] - u[i, j])^2, & i \in [0, n - 2], j \in [0, m - 2] \\
				(u[i + 1, m - 1] - u[i, m - 1])^2, & i \in [0, n - 2], j = m - 1 \\
				(u[n - 1, j + 1] - u[n - 1, j])^2, & i = n - 1, j \in [0, m - 2]
		\end{cases}
\end{align*}

\subsection{Cases for partial derivation}

We now observe where factors of $u[i, j]$ appear in the expansion of the sum
$G[u]$, to determine the different cases relevant for the differentiation. Note
that there are eight possible cases, visualized in figure
\ref{fig:gaussian_prior_derivatives}

\begin{figure}
		\centering
		\includegraphics[width=0.6\textwidth]{resources/gaussian_prior_derivatives.png}
		\caption{Regions of derivatives of gaussian prior}
		\label{fig:gaussian_prior_derivatives}
\end{figure}


\subsubsection{Case \emph{A}}

For $i = j = 0$, the factors of the sum will take on the form $u[i, j]$ for
$f(i, j)$. As such:

\begin{align*}
		\nabla_{G, A} \coloneqq \frac{\partial G[u]}{\partial u[i, j]} = 
		  \frac{\partial f(i, j)}{\partial u[i, j]}
\end{align*}

\subsubsection{Case \emph{B}}

For $1 \leq i \leq n - 2$, $j = 0$, the factors of the sum will take on the
form $u[i, j]$ for $f(i - 1, j)$ and $f(i, j)$. As such:

\begin{align*}
		\nabla_{G, B} \coloneqq \frac{\partial G[u]}{\partial u[i, j]} = 
		  \frac{\partial f(i - 1, j)}{\partial u[i, j]} 
		  + \frac{\partial f(i, j)}{\partial u[i, j]}
\end{align*}

\subsubsection{Case \emph{C}}

For $i = n - 1$, $j = 0$, the factors of the sum will take on the form $u[i,
j]$ for $f(i - 1, j)$ and $f(i, j)$. As such:

\begin{align*}
		\nabla_{G, C} \coloneqq \frac{\partial G[u]}{\partial u[i, j]} = 
		  \frac{\partial f(i - 1, j)}{\partial u[i, j]} 
		  + \frac{\partial f(i, j)}{\partial u[i, j]}
\end{align*}

\subsubsection{Case \emph{D}}

For $i = n - 1$, $1 \leq j \leq m - 2$, the factors of the sum will take on the
form $u[i, j]$ for $f(i - 1, j)$, $f(i, j - 1)$ and $f(i, j)$. As such:

\begin{align*}
		\nabla_{G, D} \coloneqq \frac{\partial G[u]}{\partial u[i, j]} = 
		  \frac{\partial f(i - 1, j)}{\partial u[i, j]} 
		  + \frac{\partial f(i, j - 1)}{\partial u[i, j]} 
		  + \frac{\partial f(i, j)}{\partial u[i, j]}
\end{align*}

\subsubsection{Case \emph{E}}

For $i = n - 1$, $j = m - 1$, the factors of the sum will take on the form
$u[i, j]$ for $f(i - 1, j)$ and $f(i, j - 1)$. As such:

\begin{align*}
		\nabla_{G, E} \coloneqq \frac{\partial G[u]}{\partial u[i, j]} = 
		  \frac{\partial f(i - 1, j)}{\partial u[i, j]}
		  + \frac{\partial f(i, j - 1)}{\partial u[i, j]}
\end{align*}

\subsubsection{Case \emph{F}}

For $1 \leq i \leq n - 2$, $j = m - 1$, the factors of the sum will take on the
form $u[i, j]$ for $f(i - 1, j)$, $f(i, j - 1)$ and $f(i, j)$. As such:

\begin{align*}
		\nabla_{G, F} \coloneqq \frac{\partial G[u]}{\partial u[i, j]} = 
		  \frac{\partial f(i - 1, j)}{\partial u[i, j]}
		  + \frac{\partial f(i, j - 1)}{\partial u[i, j]}
		  + \frac{\partial f(i, j)}{\partial u[i, j]}
\end{align*}

\subsubsection{Case \emph{G}}

For $i = 0$, $j = m -1$, the factors of the sum will take on the form $u[i, j]$
for $f(i, j - 1)$ and $f(i, j)$. As such:

\begin{align*}
		\nabla_{G, G} \coloneqq \frac{\partial G[u]}{\partial u[i, j]} = 
		  \frac{\partial f(i, j - 1)}{\partial u[i, j]}
		  + \frac{\partial f(i, j)}{\partial u[i, j]}
\end{align*}

\subsubsection{Case \emph{H}}

For $i = 0$, $1 \leq j \leq m - 2$, the factors of the sum will take on the
form $u[i, j]$ for $f(i, j - 1)$ and $f(i, j)$. As such:

\begin{align*}
		\nabla_{G, H} \coloneqq \frac{\partial G[u]}{\partial u[i, j]} = 
		  \frac{\partial f(i, j - 1)}{\partial u[i, j]}
		  + \frac{\partial f(i, j)}{\partial u[i, j]}
\end{align*}

\subsubsection{Case \emph{I}}

For $1 \leq i \leq n - 2$, $1 \leq j \leq m - 2$, the factors of the sum will
take on the form $u[i, j]$ for $f(i - 1, j)$, $f(i, j - 1)$ and $f(i, j)$. As
such:

\begin{align*}
		\nabla_{G, I} \coloneqq \frac{\partial G[u]}{\partial u[i, j]} = 
		  \frac{\partial f(i - 1, j)}{\partial u[i, j]}
		  + \frac{\partial f(i, j - 1)}{\partial u[i, j]}
		  + \frac{\partial f(i, j)}{\partial u[i, j]}
\end{align*}

\subsection{Determining the derivatives}

We now determine the partial derivatives for $u[i, j]$ of $f(i - 1, j)$, $f(i,
j - 1)$ and $f(i, j)$, using basic laws for derivatives.

\begin{align*}
		\frac{\partial f(i - 1, j)}{\partial u[i, j]} & =
		  2 \cdot u[i, j] - 2 \cdot u[i - 1, j] \\
		\frac{\partial f(i, j - 1)}{\partial u[i, j]} & =
		  2 \cdot u[i, j] - 2 \cdot u[i, j - 1] \\
		\frac{\partial f(i, j)}{\partial u[i, j]} & =
		  \begin{cases}
				  4 \cdot u[i, j] - 2 \cdot u[i + 1, j] - 2 \cdot u[i, j + 1] 
				    & i \in [0, n - 2], j \in [0, m - 2] \\
				  2 \cdot u[i, j] - 2 \cdot u[i + 1, j] 
				    & i \in [0, n - 2], j = m - 1 \\
				  2 \cdot u[i, j] - 2 \cdot u[i, j + 1] 
					& i = n - 1, j \in [0, m - 2]
		  \end{cases} \\
\end{align*}

Then the derivatives for the eight cases above follow directly:

\subsubsection{Case \emph{A}}

\begin{align*}
		\nabla_{G, A} = 
		  4 \cdot u[i, j] - 2 \cdot u[i + 1, j] - 2 \cdot u[i, j + 1]
\end{align*}

\subsubsection{Case \emph{B}}

\begin{align*}
		\nabla_{G, B} = 
		  6 \cdot u[i, j] - 2 \cdot u[i - 1, j] - 2 \cdot u[i + 1, j] - 2 \cdot u[i, j + 1]
\end{align*}

\subsubsection{Case \emph{C}}

\begin{align*}
		\nabla_{G, C} =
		  4 \cdot u[i, j] - 2 \cdot u[i - 1, j] - 2 \cdot u[i, j + 1]
\end{align*}

\subsubsection{Case \emph{D}}

\begin{align*}
		\nabla_{G, D} = 
		  6 \cdot u[i, j] - 2 \cdot u[i - 1, j] - 2 \cdot u[i, j - 1] - 2 \cdot u[i, j + 1]
\end{align*}

\subsubsection{Case \emph{E}}

\begin{align*}
		\nabla_{G, E} =
		  4 \cdot u[i, j] - 2 \cdot u[i - 1, j] - 2 \cdot u[i, j - 1]
\end{align*}

\subsubsection{Case \emph{F}}

\begin{align*}
		\nabla_{G, F} =
		  6 \cdot u[i, j] - 2 \cdot u[i, j - 1] - 2 \cdot u[i + 1, j] - 2 \cdot u[i - 1, j]
\end{align*}

\subsubsection{Case \emph{G}}

\begin{align*}
		\nabla_{G, G} = 
		  4 \cdot u[i, j] - 2 \cdot u[i, j - 1] - 2 \cdot u[i + 1, j]
\end{align*}

\subsubsection{Case \emph{H}}

\begin{align*}
		\nabla_{G, H} =
		  6 \cdot u[i, j] - 2 \cdot u[i + 1, j] - 2 \cdot u[i, j + 1] - 2 \cdot u[i, j - 1]
\end{align*}

\subsubsection{Case \emph{I}}

\begin{align*}
		\nabla_{G, I} =
		  8 \cdot u[i, j] - 2 \cdot u[i + 1, j] - 2 \cdot u[i, j + 1] - 2 \cdot u[i - 1, j] - 2 \cdot u[i, j - 1]
\end{align*}


\section{Discretization of anisotropic prior}

Starting with the anisotropic regularization term, rewritten for a $m \times n$
image:

\[
		R[u] = \abs{\nabla u}_1 = \sum_{i = 0}^{n - 1} \sum_{j = 0}^{m - 1} \abs{\nabla u[i, j]}_1
\]

We now approximate the gradient using forward differences with $\epsilon = 1$:
\[
		\nabla u[i, j] \approx \begin{bmatrix}
				u[i + 1, j] - u[i, j] \\
				u[i, j + 1] - u[i, j]
		\end{bmatrix}
\]

And as such its L1 norm:
\[
		\abs{\nabla u[i, j]}_1 \approx \abs{u[i + 1, j] - u[i, j]} + \abs{u[i, j + 1] - u[i, j]}
\]

To ensure this formula is well-defined, the right and bottom edge must be
handled specially. We assume that pixels outside the image are repeated values
of their closest analogue within the image, implying that those differences are
zero. Then:
\begin{align*}
		\abs{\nabla u[i, j]}_1 \approx \begin{cases}
				\abs{u[i + 1, j] - u[i, j]} + \abs{u[i, j + 1] - u[i, j]} & i \in [1, n - 2], j \in [1, m - 2] \\
				\abs{u[i + 1, m - 1] - u[i, m - 1]} & i \in [1, n - 2], j = m - 1 \\
				\abs{u[n - 1, j + 1] - u[n - 1, j]} & i = n - 1, j \in [1, m - 2] \\
				0 & i = n - 1, j = m - 1
		\end{cases}
\end{align*}

Thus we can discretize our regularization term as follows:
\begin{align*}
		A[u] \coloneqq 
		     & \sum_{i=0}^{n-2} \sum_{j=0}^{m-2} \left(\abs{u[i + 1, j] - u[i, j]} + \abs{u[i,j + 1] - u[i, j]}\right) \\
			 & + \sum_{i=0}^{n-2} \left(\abs{u[i + 1, m - 1] - u[i, m - 1]}\right) \\
			 & + \sum_{j=0}^{m-2} \left(\abs{u[n - 1, j + 1] - u[n - 1, j]}\right)
\end{align*}

As before, let $f(\hat{i}, \hat{j})$ denote the summand of $A[u]$ for $i =
\hat{i}, j = \hat{j}$.  That is:

\begin{align*}
		f(i, j) \coloneqq 
		\begin{cases}
				\abs{u[i + 1, j] - u[i, j]} + \abs{u[i, j + 1] - u[i, j]}, & i \in [0, n - 2], j \in [0, m - 2] \\
				\abs{u[i + 1, m - 1] - u[i, m - 1]}, & i \in [0, n - 2], j = m - 1 \\
				\abs{u[n - 1, j + 1] - u[n - 1, j]}, & i = n - 1, j \in [0, m - 2]
		\end{cases}
\end{align*}

\section{Derivative of anisotropic prior}
\label{sec:derivative_of_anisotropic}

Observe that the discretization of the anisotropic prior is structured
extremely similar to the discretization of the Gaussian prior, with the sole
exception that the square function is replaced by the absolute value. This
allows to significantly reduce the manual labour involved in deriving the
derivatives.

\subsection{Cases for partial derivation}

First, as the sum has the same structure, the exact same cases as for the
Gaussian prior will be relevant. These are, once again, shown in figure
\ref{fig:anisotropic_prior_derivatives}

\begin{figure}
		\centering
		% Yup, it's the same image as the Gaussian
		\includegraphics[width=0.6\textwidth]{resources/gaussian_prior_derivatives.png}
		\caption{Regions of derivatives of anisotropic prior}
		\label{fig:anisotropic_prior_derivatives}
\end{figure}


\subsubsection{Case \emph{A}}

For $i = j = 0$, the factors of the sum will take on the form $u[i, j]$ for
$f(i, j)$. As such:

\begin{align*}
		\nabla_{A, A} \coloneqq \frac{\partial A[u]}{\partial u[i, j]} = 
		  \frac{\partial f(i, j)}{\partial u[i, j]}
\end{align*}

\subsubsection{Case \emph{B}}

For $1 \leq i \leq n - 2$, $j = 0$, the factors of the sum will take on the
form $u[i, j]$ for $f(i - 1, j)$ and $f(i, j)$. As such:

\begin{align*}
		\nabla_{A, B} \coloneqq \frac{\partial A[u]}{\partial u[i, j]} = 
		  \frac{\partial f(i - 1, j)}{\partial u[i, j]} 
		  + \frac{\partial f(i, j)}{\partial u[i, j]}
\end{align*}

\subsubsection{Case \emph{C}}

For $i = n - 1$, $j = 0$, the factors of the sum will take on the form $u[i,
j]$ for $f(i - 1, j)$ and $f(i, j)$. As such:

\begin{align*}
		\nabla_{A, C} \coloneqq \frac{\partial A[u]}{\partial u[i, j]} = 
		  \frac{\partial f(i - 1, j)}{\partial u[i, j]} 
		  + \frac{\partial f(i, j)}{\partial u[i, j]}
\end{align*}

\subsubsection{Case \emph{D}}

For $i = n - 1$, $1 \leq j \leq m - 2$, the factors of the sum will take on the
form $u[i, j]$ for $f(i - 1, j)$, $f(i, j - 1)$ and $f(i, j)$. As such:

\begin{align*}
		\nabla_{A, D} \coloneqq \frac{\partial A[u]}{\partial u[i, j]} = 
		  \frac{\partial f(i - 1, j)}{\partial u[i, j]} 
		  + \frac{\partial f(i, j - 1)}{\partial u[i, j]} 
		  + \frac{\partial f(i, j)}{\partial u[i, j]}
\end{align*}

\subsubsection{Case \emph{E}}

For $i = n - 1$, $j = m - 1$, the factors of the sum will take on the form
$u[i, j]$ for $f(i - 1, j)$ and $f(i, j - 1)$. As such:

\begin{align*}
		\nabla_{A, E} \coloneqq \frac{\partial A[u]}{\partial u[i, j]} = 
		  \frac{\partial f(i - 1, j)}{\partial u[i, j]}
		  + \frac{\partial f(i, j - 1)}{\partial u[i, j]}
\end{align*}

\subsubsection{Case \emph{F}}

For $1 \leq i \leq n - 2$, $j = m - 1$, the factors of the sum will take on the
form $u[i, j]$ for $f(i - 1, j)$, $f(i, j - 1)$ and $f(i, j)$. As such:

\begin{align*}
		\nabla_{A, F} \coloneqq \frac{\partial A[u]}{\partial u[i, j]} = 
		  \frac{\partial f(i - 1, j)}{\partial u[i, j]}
		  + \frac{\partial f(i, j - 1)}{\partial u[i, j]}
		  + \frac{\partial f(i, j)}{\partial u[i, j]}
\end{align*}

\subsubsection{Case \emph{G}}

For $i = 0$, $j = m -1$, the factors of the sum will take on the form $u[i, j]$
for $f(i, j - 1)$ and $f(i, j)$. As such:

\begin{align*}
		\nabla_{A, G} \coloneqq \frac{\partial A[u]}{\partial u[i, j]} = 
		  \frac{\partial f(i, j - 1)}{\partial u[i, j]}
		  + \frac{\partial f(i, j)}{\partial u[i, j]}
\end{align*}

\subsubsection{Case \emph{H}}

For $i = 0$, $1 \leq j \leq m - 2$, the factors of the sum will take on the
form $u[i, j]$ for $f(i, j - 1)$ and $f(i, j)$. As such:

\begin{align*}
		\nabla_{A, H} \coloneqq \frac{\partial A[u]}{\partial u[i, j]} = 
		  \frac{\partial f(i, j - 1)}{\partial u[i, j]}
		  + \frac{\partial f(i, j)}{\partial u[i, j]}
\end{align*}

\subsubsection{Case \emph{I}}

For $1 \leq i \leq n - 2$, $1 \leq j \leq m - 2$, the factors of the sum will
take on the form $u[i, j]$ for $f(i - 1, j)$, $f(i, j - 1)$ and $f(i, j)$. As
such:

\begin{align*}
		\nabla_{A, I} \coloneqq \frac{\partial A[u]}{\partial u[i, j]} = 
		  \frac{\partial f(i - 1, j)}{\partial u[i, j]}
		  + \frac{\partial f(i, j - 1)}{\partial u[i, j]}
		  + \frac{\partial f(i, j)}{\partial u[i, j]}
\end{align*}

\subsection{Determining the derivatives}

Second, the replacement of the square function with the absolute value one will
, by the chain rule, only affect the outer derivative. Compare the following
two derivatives:

\begin{align*}
		g(x) & \coloneqq \left(f(x)\right)^2 \\
		h(x) & \coloneqq \abs*{f(x)} \\
		g'(x) & = 2 f(x) \cdot f'(x) \\
		h'(x) & = \sgn(f(x)) \cdot f(x)'
\end{align*}

As such the derivatives for the anisotropic prior follow directly from the
derivatives of the Gaussian prior, by simply replacing occurences of $2$ by the
$\sgn$ function. Some care has to be taken as the sign function, unlike
multiplication, does not distribute over addition (respectively subtraction).
Some simplifications are made using the fact that $-\sgn(x) = \sgn(-x)$ for $x
\in \mathbb{R}$.

\begin{align*}
		\frac{\partial f(i - 1, j)}{\partial u[i, j]} & =
		  \sgn(u[i, j] - u[i - 1, j]) \\
		\frac{\partial f(i, j - 1)}{\partial u[i, j]} & =
		  \sgn(u[i, j] - u[i, j - 1]) \\
		\frac{\partial f(i, j)}{\partial u[i, j]} & =
		  \begin{cases}
				  \sgn(u[i, j] - u[i + 1, j]) + \sgn(u[i, j] - u[i, j - 1])
				    & i \in [0, n - 2], j \in [0, m - 2] \\
				  \sgn(u[i, j] - u[i + 1, j])
				    & i \in [0, n - 2], j = m - 1 \\
				  \sgn(u[i, j] - u[i, j + 1])
					& i = n - 1, j \in [0, m - 2]
		  \end{cases} \\
\end{align*}

Then the derivatives for the eight cases above follow directly:

\subsubsection{Case \emph{A}}

\begin{align*}
		\nabla_{A, A} = 
		  \sgn(u[i, j] - u[i + 1, j]) + \sgn(u[i, j] - u[i, j + 1]) \\
\end{align*}

\subsubsection{Case \emph{B}}

\begin{align*}
		\nabla_{A, B} = 
		  \sgn(u[i, j] - u[i - 1, j]) + \sgn(u[i, j] - u[i + 1, j]) + \sgn(u[i, j] - u[i, j + 1])
\end{align*}

\subsubsection{Case \emph{C}}

\begin{align*}
		\nabla_{A, C} =
		  \sgn(u[i, j] - u[i - 1, j]) + \sgn(u[i, j] - u[i, j + 1])
\end{align*}

\subsubsection{Case \emph{D}}

\begin{align*}
		\nabla_{A, D} = 
		  \sgn(u[i, j] - u[i - 1, j]) + \sgn(u[i, j] - u[i, j - 1]) + \sgn(u[i, j] - u[i, j + 1])
\end{align*}

\subsubsection{Case \emph{E}}

\begin{align*}
		\nabla_{A, E} =
		  \sgn(u[i, j] - u[i - 1, j]) + \sgn(u[i, j] - u[i, j - 1])
\end{align*}

\subsubsection{Case \emph{F}}

\begin{align*}
		\nabla_{A, F} =
		  \sgn(u[i, j] - u[i, j -1]) + \sgn(u[i, j] - u[i + 1, j]) + \sgn(u[i, j] - u[i - 1, j])
\end{align*}

\subsubsection{Case \emph{G}}

\begin{align*}
		\nabla_{A, G} = 
		  \sgn(u[i, j] - u[i, j -1]) + \sgn(u[i, j] - u[i + 1, j])
\end{align*}

\subsubsection{Case \emph{H}}

\begin{align*}
		\nabla_{A, H} =
		  \sgn(u[i, j] - u[i + 1, j]) + \sgn(u[i, j] - u[i, j + 1]) + \sgn(u[i, j] - u[i, j -1])
\end{align*}

\subsubsection{Case \emph{I}}

\begin{align*}
		\nabla_{A, I} =
		  & \sgn(u[i, j] - u[i + 1, j]) + \sgn(u[i, j] - u[i, j + 1]) + \\
		  & \sgn(u[i, j] - u[i - 1, j]) + \sgn(u[i, j] - u[i, j - 1])
\end{align*}


\section{Gradient of energy term}

Now the gradient of the full energy term follows easily as:
\begin{align*}
		\nabla_u E = 
		\begin{cases}
				\nabla_{D, X} & \text{ with no regularization} \\
				\nabla_{D, X} + \lambda \nabla_{G, Y} & \text{ with Gaussian regularization} \\
				\nabla_{D, X} + \lambda \nabla_{A, Y} & \text{ with anisotropic regularization}
		\end{cases}
\end{align*}

Where, given the pixel position $(i, j)$, $\nabla_{D, X}$ is the appropriate
derivative of the data term as per section \ref{sec:derivative_of_data_term},
$\nabla_{G, Y}$ the appropriate derivative of the Gaussian prior as per section
\ref{sec:derivative_of_gaussian}, and $\nabla_{A, Y}$ the appropriate
derivative of the anisotropic prior as per section
\ref{sec:derivative_of_anisotropic} respectively.

\printbibliography

\end{document}
