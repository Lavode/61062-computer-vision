\documentclass[a4paper]{scrreprt}

% Uncomment to optimize for double-sided printing.
% \KOMAoptions{twoside}

% Set binding correction manually, if known.
% \KOMAoptions{BCOR=2cm}

% Localization options
\usepackage[english]{babel}
\usepackage[T1]{fontenc}
\usepackage[utf8]{inputenc}

% Quotations
\usepackage{dirtytalk}

% Floats
\usepackage{float}

% Enhanced verbatim sections. We're mainly interested in
% \verbatiminput though.
\usepackage{verbatim}

% Automatically remove leading whitespace in lstlisting
\usepackage{lstautogobble}

% PDF-compatible landscape mode.
% Makes PDF viewers show the page rotated by 90°.
\usepackage{pdflscape}

% Advanced tables
\usepackage{array}
\usepackage{tabularx}
\usepackage{longtable}

% Fancy tablerules
\usepackage{booktabs}

% Graphics
\usepackage{graphicx}

% Current time
\usepackage[useregional=numeric]{datetime2}

% Float barriers.
% Automatically add a FloatBarrier to each \section
\usepackage[section]{placeins}

% Custom header and footer
\usepackage{fancyhdr}

\usepackage{geometry}
\usepackage{layout}

% Math tools
\usepackage{mathtools}
% Math symbols
\usepackage{amsmath,amsfonts,amssymb}
\usepackage{amsthm}
% General symbols
\usepackage{stmaryrd}

% Utilities for quotations
\usepackage{csquotes}

% Bibliography
\usepackage[
  style=alphabetic,
  backend=biber, % Default backend, just listed for completness
  sorting=ynt % Sort by year, name, title
]{biblatex}
\addbibresource{references.bib}

\DeclarePairedDelimiter\abs{\lvert}{\rvert}
\DeclarePairedDelimiter\norm{\lVert}{\rVert}
\DeclarePairedDelimiter\floor{\lfloor}{\rfloor}

% Bullet point
\newcommand{\tabitem}{~~\llap{\textbullet}~~}

\pagestyle{plain}
% \fancyhf{}
% \lhead{}
% \lfoot{}
% \rfoot{}
% 
% Source code & highlighting
\usepackage{listings}

% SI units
\usepackage[binary-units=true]{siunitx}
\DeclareSIUnit\cycles{cycles}

\newcommand{\lecture}{61062 - Computer Vision}
\newcommand{\series}{01}
% Convenience commands
\newcommand{\mailsubject}{\lecture - Practical \series}
\newcommand{\maillink}[1]{\href{mailto:#1?subject=\mailsubject}
                               {#1}}

% Should use this command wherever the print date is mentioned.
\newcommand{\printdate}{\today}

\subject{\lecture}
\title{Practical \series}
\subtitle{Report}

\author{Michael Senn \maillink{michael.senn@students.unibe.ch} --- 16-126-880}

\date{\printdate}

% Needs to be the last command in the preamble, for one reason or
% another. 
\usepackage{hyperref}

\begin{document}
\maketitle


\setcounter{chapter}{\numexpr \series - 1 \relax}

\chapter{Practical - Report}

\section{Derivative of the data term}

Starting with the provided discretization of the data term:
\[
		\abs*{u \times k - g}^2 = \sum_{i=0}^{m-1} \sum_{j=0}^{n-1} \abs*{g[i, j] - \sum_{p=0}^{i} \sum_{q=0}^{i} k[p, q] \cdot u[i - p + 1, j - q + 1]}_2^2
\]

We simplify the data term by substituting the assigned blur kernel $k_0$:
\[
		k_0 \coloneqq \begin{bmatrix}
				\frac{1}{2} & \frac{1}{2} \\
				0 & 0
		\end{bmatrix}
\]

\begin{align*}
		D[u] & \coloneqq \abs*{u \times k_0 - g}^2 \\ 
			 & = \sum_{i=0}^{m-1} \sum_{j=0}^{n-1} \abs*{g[i, j] - \frac{1}{2} \cdot (u[i + 1, j + 1] + u[i + 1, j])}_2^2 \\
			 & = \sum_{i=0}^{m-1} \sum_{j=0}^{n-1} \left(g[i, j] - \frac{1}{2} \cdot (u[i + 1, j + 1] + u[i + 1, j])\right)^2
\end{align*}

Where the second simplification is due to the $l_2$ norm of a scalar being
equal to the absolute value of a scalar --- and the square of a real value
being equal to its absolute square. We denote this kernel-specific data term as
$D[u]$ to abbreivate notation. Let further $f(i, j)$ denote the summand:

\begin{align*}
		f(i, j) & \coloneqq \left(g[i, j] - \frac{1}{2} \cdot (u[i + 1, j + 1] + u[i + 1, j])\right)^2
\end{align*}

\subsection{Specific cases for partial derivation}

We now observe which factors of $u[i, j]$ appear in the expansion of the sum
$D[u]$, to determine the different cases relevant for the differentiation. Note
that there are four possible cases, visualized for a $m \times n$ image in
figure \ref{fig:data_term_derivatives}

\begin{figure}
		\centering
		\includegraphics[width=0.6\textwidth]{resources/data_term_derivatives.drawio.png}
		\caption{Regions of derivatives of data term}
		\label{fig:data_term_derivatives}
\end{figure}


\subsubsection{Case \emph{A}}

For $i = 0$, none of the factors of the sum will take on the form $u[i, j]$, so
the derivative is a constant $0$.


\subsubsection{Case \emph{B}}

For $1 \leq i \leq n-1$, $j = 0$, the factors of the sum will take on the form
$u[i, j]$ only for $f(i - 1, j)$. As such:

\begin{align*}
		\frac{\partial D[u]}{\partial u[i, j]} = \frac{\partial f(i - 1, j)}{\partial u[i, j]}
\end{align*}


\subsubsection{Case \emph{C}}

For $1 \leq i \leq n-1$, $j = m - 1$, the factors of the sum will take on the form
$u[i, j]$ only for $f(i - 1, j - 1)$. As such:

\begin{align*}
		\frac{\partial D[u]}{\partial u[i, j]} = \frac{\partial f(i - 1, j - 1)}{\partial u[i, j]}
\end{align*}


\subsubsection{Case \emph{D}}

For $1 \leq i \leq n-1$, $1 \leq j \leq m - 2$, the factors of the sum will
take on the form $u[i, j]$ both for $f(i - 1, j)$ as well as for $f(i - 1, j -
1)$. As such:

\begin{align*}
		\frac{\partial D[u]}{\partial u[i, j]} = \frac{\partial f(i - 1, j)}{\partial u[i, j]} + \frac{\partial f(i - 1, j - 1)}{\partial u[i, j]}
\end{align*}

\subsection{Determining the derivatives}

We now determine the partial derivatives for $u[i, j]$ at $f(i - 1, j - 1)$ and
$f(i - 1, j)$, using basic laws for derivatives.

\begin{align*}
		\frac{\partial f(i - 1, j)}{\partial u[i, j]} & =
		\frac{1}{2} \cdot u[i, j + 1] + \frac{1}{2} \cdot u[i, j] - g[i - 1, j] \\
		\frac{\partial f(i - 1, j - 1)}{\partial u[i, j]} & =
		\frac{1}{2} \cdot u[i, j - 1] + \frac{1}{2} \cdot u[i, j] - g[i - 1, j - 1]
\end{align*}

Then the derivatives for the four cases above follow directly:

\subsubsection{Case \emph{A}}

For $1 \leq i \leq n, j = 0$:

\[
		\frac{\partial D[u]}{\partial u[i, j]} = 0
\]

\subsubsection{Case \emph{B}}

For $1 \leq i \leq n - 1, j = 0$:

\[
		\frac{\partial D[u]}{\partial u[i, j]} =
		\frac{1}{2} \cdot u[i, j + 1] + \frac{1}{2} \cdot u[i, j] - g[i - 1, j],
\]

\subsubsection{Case \emph{C}}

For $1 \leq i \leq n - 1, j = m - 1$:

\[
		\frac{\partial D[u]}{\partial u[i, j]} =
		\frac{1}{2} \cdot u[i, j - 1] + \frac{1}{2} \cdot u[i, j] - g[i - 1, j - 1]
\]

\subsubsection{Case \emph{D}}

For $1 \leq i \leq n - 1, 1 \leq j \leq m - 2$:

\[
		\frac{\partial D[u]}{\partial u[i, j]} =
						  \frac{1}{2} \cdot u[i, j + 1] + \frac{1}{2} \cdot u[i, j] - g[i - 1, j] + \frac{1}{2} \cdot u[i, j - 1] + \frac{1}{2} \cdot u[i, j] - g[i - 1, j - 1]
\]

\printbibliography

\end{document}
